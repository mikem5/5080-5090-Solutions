

{\bf 09.22} a) First we find $\tau(\theta)$, in our case it is $\mu$ so we compute $(\tau'(\theta))^2 = 1$. Next we turn
our attention to the denominator of our CRLB: \\
\begin{align*}
	f(x;\mu,9) & =	\frac{1}{\sqrt{2 \pi 9}} e^{-\frac{(x - \mu)^{2}}{(2)(9)}} \\
	\ln f & = \ln \left( \frac{1}{\sqrt{2 \pi 9}}\right) +  - \frac{(x - \mu)^{2}}{(2)(9)} \\
	\frac{d \ln f}{d \mu} & = \frac{(x - \mu)}{(9)} \\
	\frac{d^2 \ln f}{d \mu^2} & =  - \frac{1}{9} \\
\end{align*} 

We would normally need to take the expectation here but a constant will yield a constant so we may simply continue
and compute the rest of our CRLB. \\
\begin{align*}
	Var(T) & \geq \frac{1}{ - n \left( \frac{-1}{9} \right) } \\
		& \geq \frac {9}{n} 
\end{align*}
Which is our CRLB. \\

b) Since $Var(\overline{X}) = \frac{\sigma^2}{n}$ we know that this MLE is a UMVUE since it is exactly our CRLB. \\

c) We have to find the 95th percentile first,  $\Pr ( X \leq x_{.95} )$. Since we are normal, we just transform to
a standard normal, then find the actual value using a table. $\Phi \left( \frac{x_{.95} - \mu}{3} \right) $ and then 
$\frac{x_{.95} - \mu}{3} = 1.645$ so $x_{.95} = 4.935 + \mu$. Due to invariance we can set our 
$\tau(\hat{\theta)} = 4.935 + \hat{\mu}$ where $\hat{\mu} = \overline{X}$, our MLE of $\mu$. The variance of $\hat{\theta}$
is clearly a UMVUE since it is the CRLB as well, this being due to the  rules of variances having any constant zeroed out, 
leaving us with our normal $\mu$. \\



